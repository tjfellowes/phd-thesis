\chapter{Ch-bonding as a supramolecular tool}
\section{Introduction}


\section{Experimental procedures}

\subsection{Synthetic methods}
\subsubsection{Preparation of 2-nitroterephthalic acid (\refcmpd{nitro-diacid})}
%TF-7-53
%KNOWN
Sulfuric acid (98\%, 20~mL) and nitric acid (15~mL) were slowly mixed and cooled to 20\degree C. 
To this was added portionwise terephthalic acid (5.079~g, 30.57~mmol), and the mixture was then refluxed for 30 minutes. 
It was then cooled to room temperature, and poured over 200~mL of crushed ice. 
The resulting slurry was allowed to melt, then filtered, affording \cmpd{nitro-diacid} as an off white powder (5.409~g, 84\%).

\footnotesize\paragraph{}

\ce{^{1}H} NMR 400~MHz, \ce{\emph{d}6}-DMSO) $\delta$ ppm 
8.37 (s, 1~H), 8.27 (d, J=7.8~Hz, 1~H), 7.94 (d, J=7.8~Hz, 1~H).

\normalsize

\subsubsection{Preparation of 2-nitroterephthalic acid dimethyl ester (\refcmpd{nitro-diacid-me-ester})}
%TF-7-61
%KNOWN
2-Nitroterephthalic acid (1.007~g, 4.770~mmol) was dissolved in methanol (20~mL), and to this was added concentrated sulfuric acid (5~mL). The mixture was refluxed for 24~h, then cooled, diluted with water (100~mL) and neutralised with saturated sodium bicarbonate solution. The resulting precipitate was filtered off and dried, affording the ester \cmpd{nitro-diacid-me-ester} as a colourless solid (950.4~mg, 83\%).

\footnotesize\paragraph{}

\ce{^{1}H} NMR 400~MHz, \ce{\emph{d}6}-DMSO) $\delta$ ppm 


\normalsize

\subsubsection{Preparation of 2-aminoterephthalic acid (\refcmpd{aniline-diacid})}
%TF-7-57
%KNOWN
Nitrobenzene \cmpd{nitro-diacid} (2.1241~g, 9.6057~mmol) was dissolved in ethanol (70~mL), and the flask was evacuated and backfilled with nitrogen twice. 
Palladium on carbon (10\%, 176.2~mg) was then added, and the flask was evacuated and backfilled with hydrogen twice. 
The mixture was stirred under hydrogen for 18~h, then again placed under nitrogen, and heated to reflux. 
The solution was filtered while hot, and the filtrate was evaporated to afford \cmpd{aniline-diacid} as a yellow powder (1.3570~g, 75\%).

\footnotesize\paragraph{}

\ce{^{1}H} NMR (400~MHz, \ce{\emph{d}6}-DMSO) $\delta$ ppm 
7.74 (d, J=8.2~Hz, 1~H), 7.36 (s, 1~H), 6.99 (d, J=8.2~Hz, 1~H)
\normalsize