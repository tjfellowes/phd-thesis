
\addtotoc{Abstract}  % Add the "Abstract" page entry to the Contents
\abstract{
\addtocontents{toc}{\vspace{1em}}  % Add a gap in the Contents, for aesthetics
Chalcogen bonding (Ch-bonding) is a non-covalent attractive interaction between a Lewis base and a positively polarised group 6 element, typically sulfur, selenium, or tellurium.
It has been recognised as complementary to the related phenomena of hydrogen bonding and halogen bonding, and it has potential applications in fields as diverse as crystal engineering, anion sensing, and medicinal chemistry.

This work was originally motivated by the design of a Ch-bonding DNA binder.
Throughout the design and synthesis of this molecule, we investigated some simpler systems to gain a better understanding of the Ch-bond.
X-ray diffraction analysis of co-crystals with Lewis bases afforded data which showed the Ch-bond length was inversely related to the electron density at the selenium atom.
We also observed a lengthening of the endocyclic bond opposite to the Ch-bond, indicative of partial covalent character of the interaction.
Experimental electron density was analysed within the QTAIM framework, which showed that critical points associated with Ch-bonds actually bear more similarity to closed-shell, electrostatic interactions.
The unique NMR properties of the \ce{^{77}Se} nucleus were used to investigate Ch-bonding through solid state NMR.\@
Due to the large degree of polarisation of the selenium, an enormous chemical shift anisotropy is observed.
Measurement of the chemical shift tensor in a single crystal of a Ch-bonded complex showed that this is due to the approach of the Lewis base.

In a departure from selenium chemistry, we observed a \ce{O\cdots O} short contact in a \textit{o}-nitro-O-aryl oxime which showed characteristics consistent with a Ch-bond.
This piqued our interest, as oxygen is usually considered to be insufficiently polarisable to form Ch-bonds.
We prepared and studied a series of derivatives, and found sufficient evidence to claim Ch-bonding is occurring in this system.
This may have implications for our understanding of reactive oxygen species such as peroxides and nitrate radicals.
}