
\addtotoc{Abstract}  % Add the "Abstract" page entry to the Contents
\abstract{
\addtocontents{toc}{\vspace{1em}}  % Add a gap in the Contents, for aesthetics

Chalcogen bonding (Ch-bonding) is a non-covalent attractive interaction between a Lewis base and a positively polarised group 6 element, typically sulfur, selenium, or tellurium.
It has been recognised as complementary to the related phenomena of hydrogen bonding and halogen bonding, and it has potential applications in fields as diverse as crystal engineering, anion sensing, and medicinal chemistry.
In spite of these promising applications, there is a lack of experimental evidence regarding the origin of Ch-bonding, and detailed characterisation of Ch-bonded systems.

This work was originally motivated by the design of a Ch-bonding DNA binder, inspired by the bis-benzimidazole skeleton of the Hoechst compounds, but incorporating a Ch-bond donor instead of an H-bond donor.
Throughout the design and synthesis of this molecule, we investigated some simpler systems to gain a better understanding of the Ch-bond.
A number of derivatives of the drug ebselen were prepared, and crystallised with a variety of Lewis bases, in order to model the interactions present in the DNA binder.
To our delight, each combination of Ch-bond donor and Lewis base produced crystals suitable for x-ray diffraction analysis, and the resulting structural data was analysed to afford a Hammett relationship, in which the Ch-bond length was related in an inverse sense to the electron density at the selenium atom.
We also observed a lengthening of the endocyclic bond opposite to the Ch-bond, which we interpreted as partial occupation of the $ \sigma^{\star} $(\ce{Se-N}) anti-bonding orbital, indicative of partial covalent character of the interaction.
Some derivatives afforded data of sufficient quality that the experimental electron density could be determined through multipole refinement.
This was analysed withing the QTAIM framework, which showed that bond critical points associated with Ch-bonds bear more similarity to closed-shell, electrostatic interactions, suggesting that the covalent character, although present, is secondary in importance.
The unique NMR properties of the \ce{^{77}Se} nucleus were used to investigate Ch-bonding through solid state NMR.\@
Due to the large degree of polarisation of the selenium, an enormous chemical shift anisotropy is observed.
Measurement of the chemical shift tensor in a single crystal of a Ch-bonded complex showed that this is due to the approach of the Lewis base.

In a departure from selenium chemistry, we observed a \ce{O\cdots O} short contact in a \textit{o}-nitro-O-aryl oxime which showed characteristics consistent with a Ch-bond.
This piqued our interest, as oxygen is usually considered to be insufficiently polarisable to form Ch-bonds.
We prepared and studied a series of derivatives, and found we were able to manipulate this purported Ch-bond in a manner consistent with our previous characterisation of Ch-bonded systems.
This may have implications for our understanding of reactive oxygen species such as peroxides and nitrate radicals.



}